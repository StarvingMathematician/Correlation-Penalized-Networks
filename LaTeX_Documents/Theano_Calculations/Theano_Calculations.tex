\documentclass[11pt, oneside]{article}   	% use "amsart" instead of "article" for AMSLaTeX format
\usepackage{geometry}                		% See geometry.pdf to learn the layout options. There are lots.
\geometry{letterpaper}                   		% ... or a4paper or a5paper or ... 
%\geometry{landscape}                		% Activate for for rotated page geometry
%\usepackage[parfill]{parskip}    		% Activate to begin paragraphs with an empty line rather than an indent
\usepackage{graphicx}				% Use pdf, png, jpg, or eps� with pdflatex; use eps in DVI mode
								% TeX will automatically convert eps --> pdf in pdflatex		
\usepackage{amssymb}

\usepackage{amsmath}
\usepackage{amsfonts}
\usepackage{bbm}
\usepackage{hyperref}

\title{Theano Computations}
\author{Jonathan Simon}
%\date{}							% Activate to display a given date or no date

\begin{document}
\maketitle
%\section{}
%\subsection{}

\section*{MLP Activation Layer}

The hidden Layer in Theano's MLP is computed as: $X \rightarrow f(XW+b) = A$, implying that the the activation vectors are rows, and the batch-wide unit activation are columns. For an activation layer with $d$ hidden units and a minibatch of size $t$, the activation layer is given by the matrix:

\begin{gather} 
A = 
\begin{bmatrix}
    a_{11} & \dots & a_{1d} \\
    \vdots  &          & \vdots \\
    a_{t1} & \dots & a_{td}\\
\end{bmatrix}
= 
\begin{bmatrix}
    \vert  		   &          & \vert \\
    u_{1}		   & \dots & u_{d} \\
    \vert  		   &          & \vert \\
%    \vert  		   &          & \vert \\
%    \text{unit}_{1}   & \dots & \text{unit}_{d} \\
%    \vert  		   &          & \vert \\
\end{bmatrix}
= 
\begin{bmatrix}
    \rule{.75cm}{0.5pt}  		   & o_{1}          & \rule{.75cm}{0.5pt} \\
    & \vdots & \\
    \rule{.75cm}{0.5pt} 		   & o_{t}         & \rule{.75cm}{0.5pt} \\
%    \rule{.75cm}{0.5pt}  		   & \text{obs}_{1}          & \rule{.75cm}{0.5pt} \\
%    & \vdots & \\
%    \rule{.75cm}{0.5pt} 		   & \text{obs}_{t}         & \rule{.75cm}{0.5pt} \\
\end{bmatrix}
\end{gather}

\begin{flalign*}
& \text{where:} \\
& u_i = t\text{-dimensional timecourse of the } i^{\text{th}} \text{ unit} \\
& o_i = d\text{-dimensional state of the activation layer at the } i^{\text{th}} \text{ observation} \\
\end{flalign*}

%\text{where}\\
%u_i = t \text{-dimensional timecourse of the}% $i^{\text{th}}$ unit}

%u_i = t \text{-dimensional timecourse of the}% $i^{\text{th}}$ unit}
%o_i = d\text{-dimensional state of the activation layer corresponding to the $i^{\text{th}}$ observation}

%where $u_i = t$-dimensional timecourse of the $i^{\text{th}}$ unit
%and $o_i = d$-dimensional state of the activation layer corresponding to the $i^{\text{th}}$ observation

\subsection*{Covariance Computation}

We can then compute the between-unit covariance using the formulas provided on the wikipedia pages:

\begin{enumerate}
  \item \url{https://en.wikipedia.org/wiki/Sample_mean_and_covariance}
  \item \url{https://en.wikipedia.org/wiki/Estimation_of_covariance_matrices}
  \item \url{https://en.wikipedia.org/wiki/Covariance}
\end{enumerate}

\noindent
%Consider the definition of sample covariance for a pair of units $u_j$ and $u_k$, represented as $t$-dimensional column vectors:
Consider the definition of covariance for a pair of random variables $U_j$ and $U_k$ representing the idealized scalar activations of the $j^\text{th}$ and $k^\text{th}$ hidden units respectively:
\begin{gather}
\operatorname{cov}(U_j,U_k) = \operatorname{E}\left[(U_j-\operatorname{E}[U_j])(U_k-\operatorname{E}[U_k])\right]
%\operatorname{cov}(u_j,u_k) = \frac{1}{t-1}\bigg(\big(u_j-\operatorname{E}[u_j]\big)^\top\big(u_k-\operatorname{E}[u_k]\big)\bigg)
\end{gather}

\noindent
Substituting the \textbf{random variables} $U_j$ and $U_k$ for the \textbf{column vectors} $u_j$ and $u_k$ of scalar activation observations:
\begin{gather}
\operatorname{cov}(u_j,u_k) = \frac{1}{t-1}\bigg(\big(u_j-\mathbbm{1}\overline{u}_j\big)^\top\big(u_k-\mathbbm{1}\overline{u}_k\big)\bigg)
\intertext{where the \textbf{scalar} $\overline{u}_*$ is defined as:}
\begin{aligned}
\overline{u}_* &= \frac{1}{t} \sum_{i=1}^t \big[u_{*}\big]_{i} & \Bigg( = \frac{1}{t} \sum_{i=1}^t \big[o_{i}\big]_{*} \Bigg) \\
\end{aligned}\notag
\end{gather}

\noindent
We can then extend this calculation to all pairs of units in the hidden layer to compute the covariance matrix:
\begin{gather}
\Sigma = \frac{1}{t-1}\bigg(\big( A -  \mathbbm{1} \bar{o} \big)^\top \big( A -  \mathbbm{1} \bar{o} \big)\bigg)
\intertext{where the \textbf{row vector} $\overline{o}$ is defined as:}
\begin{aligned}
\overline{o} &= \frac{1}{t} \sum_{i=1}^t o_{i}\\
\end{aligned}\notag
\end{gather}

\subsection*{Correlation Computation}

%Is there an efficient way to do this? Will it actually help the algorithm if I do do it? Hopefully not, since keeping the magnitude information helps keep activations in the linear regime (if I do it prior to the nonlinearity as with batch normalization). But on the other hand, large-magnitude units will result in high-variance estimates of the associated covariances, which makes the method noisier. Hmm...\\

We can also compute the between-unit pearson correlation using the formulas provided at:
\begin{enumerate}
  \item \url{http://www.johndcook.com/blog/2008/11/05/how-to-calculate-pearson-correlation-accurately/}
  \item \url{https://en.wikipedia.org/wiki/Pearson_product-moment_correlation_coefficient}
\end{enumerate}

\noindent
Consider the definition of correlation for a pair of random variables $U_j$ and $U_k$ representing the idealized scalar activations of the $j^\text{th}$ and $k^\text{th}$ hidden units respectively:
\begin{gather}
\rho_{jk} = \dfrac{\operatorname{cov}(U_j,U_k)}{\sigma_j \sigma_k} = \dfrac{\operatorname{E}\left[(U_j-\operatorname{E}[U_j])(U_k-\operatorname{E}[U_k])\right]}{\sqrt{\operatorname{E}\left[(U_j-\operatorname{E}[U_j])^2\right]}\sqrt{\operatorname{E}\left[(U_k-\operatorname{E}[U_k])^2\right]}}
\end{gather}

\noindent
Substituting the \textbf{random variables} $U_j$ and $U_k$ for the \textbf{column vectors} $u_j$ and $u_k$ of scalar activation observations:
\begin{gather}
%r_{jk} = \dfrac{1}{t-1}\Bigg(\dfrac{\big(u_j-\mathbbm{1}\overline{u}_j\big)^\top\big(u_k-\mathbbm{1}\overline{u}_k\big)}{\sqrt{\big(u_j-\mathbbm{1}\overline{u}_j\big)^\top\big(u_j-\mathbbm{1}\overline{u}_j\big)}\sqrt{\big(u_k-\mathbbm{1}\overline{u}_k\big)^\top\big(u_k-\mathbbm{1}\overline{u}_k\big)}}\Bigg)
r_{jk} = \dfrac{\big(u_j-\mathbbm{1}\overline{u}_j\big)^\top\big(u_k-\mathbbm{1}\overline{u}_k\big)}{\sqrt{\big(u_j-\mathbbm{1}\overline{u}_j\big)^\top\big(u_j-\mathbbm{1}\overline{u}_j\big)}\sqrt{\big(u_k-\mathbbm{1}\overline{u}_k\big)^\top\big(u_k-\mathbbm{1}\overline{u}_k\big)}}
\intertext{where the \textbf{scalar} $\overline{u}_*$ is defined as:}
\begin{aligned}
\overline{u}_* &= \frac{1}{t} \sum_{i=1}^t \big[u_{*}\big]_{i} & \Bigg( = \frac{1}{t} \sum_{i=1}^t \big[o_{i}\big]_{*} \Bigg) \\
\end{aligned}\notag
\end{gather}

\noindent
Note that the normalizing factor of $\frac{1}{t-1}$ is absent in the above expression because it is canceled by the two occurances of $\sqrt{\frac{1}{t-1}}$ in the denominator.\\\\

\noindent
Define the sample variance $s_*$ as:
\begin{gather}
s_* = \sqrt{\frac{1}{t-1}\big(u_*-\mathbbm{1}\overline{u}_*\big)^\top\big(u_*-\mathbbm{1}\overline{u}_*\big)}
\end{gather}

\noindent
Then we can compute the \textbf{row vector} $s$ as:
\begin{gather}
s =
\begin{bmatrix}
    s_{1} & \dots & s_{d}
\end{bmatrix}
= \sqrt{\frac{1}{t-1}\bigg(\mathbbm{1}^\top \Big(\big( A -  \mathbbm{1} \bar{o} \big) \odot \big( A -  \mathbbm{1} \bar{o} \big)\Big)\bigg)}
\end{gather}

%Note further that we can compute a row vector such that the $*^\text{th}$ term is $\big(u_*-\mathbbm{1}\overline{u}_*\big)^\top\big(u_*-\mathbbm{1}\overline{u}_*\big)$ as:

\noindent
Using the definition of the covariance matrix from the previous section, we can then write the correlation matrix as:
\begin{gather}
\rho = \dfrac{\Sigma}{\big(\mathbbm{1}s\big)\odot\big(\mathbbm{1}s\big)^\top} = \dfrac{\Sigma}{\big(\mathbbm{1}s\big)^{\top}\odot\big(\mathbbm{1}s\big)}
\end{gather}
\noindent
where the division is performed element-wise.\\

\noindent
Additional optimizations which can be performed in the Theano code include:
\begin{enumerate}
  \item Removing the $\frac{1}{t-1}$ scaling factors, since they are present in both numerator and denominator
  \item Removing the square roots, since the final step involves element-wise squaring
  \item Replacing the multiplications by $\mathbbm{1}$ with broadcasting and axis-wise summation 
  \item Replacing the Hadamard products with element-wise squaring
\end{enumerate}

\noindent
\textbf{Update}: Implemented optimization 3 and 4, but optimizations 1 and 2 don't speed things up enough to make it worth the confusion.
\end{document}  